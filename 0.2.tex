\question{
    根据算符$\nabla$的微分性与矢量性,推导
    \equa{
        \boldsymbol{
        \nabla({A}\cdot {B}) = 
           {B}\times(\nabla\times {A}) + ({B}\cdot \nabla){A} 
         + {A}\times(\nabla\times {B}) + ({A}\cdot \nabla){B}.
        }
        \label{0.2_q1}
    }
}
    
    我们在笛卡尔坐标中使用分量的形式\footnote{采用爱因斯坦求和约定。}来证明(\ref{0.2_q1}),即证明
    \begin{equation}
        \boldsymbol{
        \left[\nabla({A}\cdot {B})\right]}_i \boldsymbol{= 
         \left[{B}\times(\nabla\times {A}) + ({B}\cdot \nabla){A} 
        + {A}\times(\nabla\times {B}) + ({A}\cdot \nabla){B}\right]}_i.
        \label{0.2_q1_i}
    \end{equation}
    根据算符$\nabla$的微分性与矢量性,直接写出(\ref{0.2_q1_i})左右两边并作比较
    \begin{equation}
        \begin{aligned}
            \mathrm{LHS.} 
            & = \partial_i (A_j B_j) \\
            & = B_j \partial_i A_j + A_j \partial_i B_j, \\
        \end{aligned}
    \end{equation}
    \begin{equation}
        \begin{aligned}
            \mathrm{RHS.} 
            & = \varepsilon_{ijk} B_j (\nabla\times \boldsymbol{A})_k + (B_j \partial_j)A_i + \varepsilon_{ijk} A_j (\nabla\times \boldsymbol{B})_k + (A_j \partial_j)B_i \\
            & = \varepsilon_{ijk} B_j \varepsilon_{klm}\partial_l A_m + (B_j \partial_j)A_i + \varepsilon_{ijk} A_j \varepsilon_{klm}\partial_l B_m + (A_j \partial_j)B_i \\
            & = (\delta_{il}\delta_{jm}-\delta_{im}\delta_{jl})B_j\partial_l A_m + B_j \partial_j A_i + (\delta_{il}\delta_{jm}-\delta_{im}\delta_{jl})A_j\partial_l B_m + A_j \partial_j B_i \\
            & = (B_j \partial_i A_j - B_j \partial_j A_i) + B_j \partial_j A_i + (A_j \partial_i B_j - A_j \partial_j B_i) + A_j \partial_j B_i \\
            & = B_j \partial_i A_j + A_j \partial_i B_j \\
            & = \mathrm{LHS.} 
        \end{aligned}
    \end{equation}
    可见(\ref{0.2_q1_i})左边等于右边,(\ref{0.2_q1_i})得证,则(\ref{0.2_q1})自然成立。证毕。
    
\question{
    根据算符$\nabla$的微分性与矢量性,推导
    \equa{
        \boldsymbol{{A}\times(\nabla\times {A}) = }\frac{1}{2}\nabla A^2 \boldsymbol{- ({A}\cdot \nabla){A}.}
        \label{0.2_q2}
    }
}
    
    可以直接使用(\ref{0.2_q1})来证明(\ref{0.2_q2})。在(\ref{0.2_q1})中令$\boldsymbol{B=A}$,得
    \begin{equation}
        \boldsymbol{\nabla({A}\cdot {A}) = }2\boldsymbol{{A}\times(\nabla\times {A}) + }2\boldsymbol{({A}\cdot \nabla){A}}
        \label{0.2_q1_2}
    \end{equation}
    将(\ref{0.2_q1_2})两边乘以$\frac{1}{2}$,并移项即得
    \begin{equation}
        \boldsymbol{{A}\times(\nabla\times {A}) = }\frac{1}{2}\nabla A^2 \boldsymbol{- ({A}\cdot \nabla){A}.}
    \end{equation}
    此即(\ref{0.2_q2}),证毕。

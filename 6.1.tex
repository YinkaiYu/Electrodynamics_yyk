\question{证明牛顿定律在伽利略变换下是协变的,麦克斯韦方程在伽利略变换下不是协变的。}

    方便起见,以一维伽利略变换为例。
    \equa{
        \begin{cases}
            x'=x-vt \\
            t'=t \\
        \end{cases}
        \label{6.1x伽利略变换}
    }
    由(\ref{6.1x伽利略变换})得
    \equa{u'=\dot{x'}=\dot{x}-v=u-v}
    此即牛顿第一定律:在不同的惯性系中,作惯性运动的物体始终作惯性运动。
    
    下面证牛顿第二定律是协变的。通常来讲,力的定义式就是
    \equa{F=m\dd{u}{t}}
    但是在\nota{\Sigma'}系中,上式却是需要我们证明的协变方程,因此不能用上式定义\nota{F'},而要等价地定义为
    \equa{F'=F}
    \equa{m'=m}
    则在在\nota{\Sigma'}系中有
    \equa{F'=F=m\dd{u}{t}=m\dd{u-v}{t}=m'\dd{u'}{t'}}
    于是得证牛顿定律在不同惯性系中有相同的形式。
    
    对于麦克斯韦方程,采用反证法。假设麦克斯韦方程在伽利略变换下协变,则可以作如下推论。在真空中
    \equa{\xuandu\e=-\pd{\b}{t}\label{6.1x1}}
    \equa{\sandu\e=0\label{6.1x2}}
    \equa{\xuandu\b=\muz\eps_0\pd{\e}{t}\label{6.1x3}}
    对\neq{6.1x1}两边取旋度,并结合\neq{6.1x2}\neq{6.1x3}得
    \equa{\xuandu\kuohao{\xuandu\e}=-\pd{}{t}\xuandu\b}
    \equa{\tidu\kuohao{\sandu\e}-\tidu^2\e=-\muz\eps_0\ppd{\e}{t}}
    即
    \equa{\tidu^2\e-\muz\eps_0\ppd{\e}{t}=0}
    此即电场的波动方程,其群速度为
    \equa{u=\frac{1}{\sqrt{\muz\eps_0}}\label{6.1x群速1}}
    如果麦克斯韦方程组在伽利略变换下协变,则上述推导亦是协变的,电场的波动方程也协变,有
    \equa{\tidu'^2\e'-\muz\eps_0\ppd{\e'}{t'}=0}
    其群速度为
    \equa{u'=\frac{1}{\sqrt{\muz\eps_0}}\label{6.1x群速2}}
    取群速度方向为x轴作v-boost伽利略变换,则要求
    \equa{u'=u-v\label{6.1x群速伽利略}}
    \neq{6.1x群速1}\neq{6.1x群速2}\neq{6.1x群速伽利略}矛盾,原假设不能成立,故麦克斯韦方程在伽利略变换下不是协变的。
    

\question{设有两根互相平行的尺,在各自静止的参考系中长度均为\nota{l_0},它们以相同速率v相对于某一参考系运动,但运动方向相反,且平行于尺子。求站在一根尺上测量另一根尺的长度。}

    取观察者所站的尺子的运动方向为x轴正方向,记地面系为\nota{\Sigma},观察者所站的尺子的静止参考系为\nota{\Sigma'},被观察的尺子的静止参考系为\nota{\Sigma''}。在三个参考系的时空原点处,两把尺子的左端对齐。
    
    考虑这样一个世界点(事件),在\nota{\Sigma'}系中,观察者在\nota{x'^0=0}时刻
    \footnote{
        记洛伦兹矢量
        \equa{x^\mu=
        \begin{pmatrix}
        \i ct & \x
        \end{pmatrix}\trans
        }
        相应的度规为
        \equa{g_{\mu\nu}=\diag{1&1&1&1}}
        洛伦兹变换张量
        \equa{
            \zhang{\Lambda}{\mu}{\nu}=
            \mat{
                \gamma & -\i\beta\gamma & 0 & 0 \\
                \i\beta\gamma & \gamma & 0 & 0 \\
                0 & 0 & 1 & 0 \\
                0 & 0 & 0 & 1 \\
            }
        }
        特别地,此时有
        \equa{\Lambda\trans=\Lambda^{-1}\label{6.1x正交}}
    }
    测量待测尺右端的空间坐标\nota{x'^1}。同时知道了\nota{x'^0=0}时刻的待测尺两端坐标,观察者就可以在\nota{\Sigma'}系中读出尺长
    \equa{l=x'^1-0=x'^1\label{6.1xb1}}
    “观察者在\nota{x'^0=0}时刻测量待测尺右端的空间坐标\nota{x'^1}”记为世界点
    \equa{x'^\mu=\mat{0&x'^1&0&0\\}\trans}
    有洛伦兹变换\footnote{采用爱因斯坦求和约定。}(\nota{x^1}-boost)
    \equa{x'^\mu=\zhang{\Lambda}{\mu}{\nu}x^\nu\label{6.1x洛伦兹1}}
    \equa{x''^\sigma=\zhang{\kuohao{\Lambda^{-1}}}{\sigma}{\nu}x^\nu\label{6.1x洛伦兹2}}
    其中
    \equa{\beta=\frac{v}{c}\qquad\gamma=\frac{1}{\sqrt{1-\beta^2}}\label{6.1xb2}}
    由\neq{6.1x洛伦兹1}\neq{6.1x洛伦兹2}\neq{6.1x正交}得
    \equa{x''^\sigma=\zhang{\kuohao{\Lambda\trans}}{\sigma}{\nu}\zhang{\kuohao{\Lambda\trans}}{\nu}{\mu}x'^\mu}
    其中具体地有
    \equa{x''^1=\kuohao{1+\beta^2}\gamma^2x'^1\label{6.1xb3}}
    而在\ckx{''}中显然该世界点同样出现在尺子右端,即
    \equa{x''^1=l_0\label{6.1xb4}}
    由\neq{6.1xb1}\neq{6.1xb2}\neq{6.1xb3}\neq{6.1xb4}得
    \equa{l=\frac{1-v^2/c^2}{1+v^2/c^2}l_0}


\question{参考教材,}

    \subquestion{推导稳定超导电流满足的方程\equa{\nabla^2\j_s=\frac{1}{\lambda_L^2}\j_s}其中\nota{\lambda_L}是伦敦穿透深度。}
    
        伦敦第二方程
        \equa{\xuandu\j_s=-\alpha\b\label{3.2_1}}
        对(\ref{3.2_1})取旋度并利用恒定条件
        \equa{\sandu\j_s=0}
        以及麦克斯韦方程组
        \equa{\xuandu\b=\mu_0\j_s}
        进行化简
        \equa{\xuandu\kuohao{\xuandu\j_s}=-\alpha\kuohao{\xuandu\b}}
        \equa{\nabla\kuohao{\sandu\j_s}-\nabla^2\j_s=-\alpha\mu_0\j_s}
        \equa{\nabla^2\j_s=\frac{1}{\lambda_L^2}\j_s\label{3.2_2}}
        其中
        \equa{\lambda_L=\frac{1}{\sqrt{\mu_0\alpha}}}
        
    
    \subquestion{若超导体充满在\nota{z<0}的空间,并已知在\nota{z=0}处有稳定超导电流密度为\nota{\j_s(0)},方向沿\nota{x}方向,求超导体中的稳定超导电流密度。}
    
        由于系统具有x,y方向连续平移对称性
        \equa{\pd{^2}{x^2}=\pd{^2}{y^2}=0}
        则(\ref{3.2_2})转化为
        \equa{\frac{\d^2}{\d z^2}\j_s(z)=\frac{1}{\lambda_L^2}\j_s(z)\qquad(z<0)}
        其具有物理意义的解为
        \equa{\j_s(z)=\j_s(0)\mathrm{e}^{-z/\lambda_L}}
    
\question{半径为\nota{a}、处于理想迈斯纳态的超导球附近,距球心为\nota{d(d>a)}处有一沿球径方向的磁偶极子\nota{\m}。证明:\nota{\m}的镜像为\nota{\m'=-\kuohao{a/d}^3\m},位置在球内\nota{z={a^2}/{d}}处。}

    只需验证磁场解满足边界条件
    \equa{B_n=0\label{3.2_bianj}}
    由于不存在自由电流,可以引入磁标势来描述磁场,磁偶极子的磁标势解为
    \equa{\varphi_m^{(1)}=\frac{\m\cdot\bm{R}}{4\pi R^3}}
    \equa{\b^{(1)}=\nabla\varphi_m^{(1)}}
    其中
    \equa{\bm{R}=\bm{x}-\bm{x'}}
    
    如果\nota{\m}的镜像为\nota{\m'=-\kuohao{a/d}^3\m},位置在球内\nota{z={a^2}/{d}}处,则以磁偶极子所沿的球径方向取极轴,建立球坐标系,外部空间中的磁标势为
    \equa{
        \begin{aligned}
            \varphi_m
            &=
            \frac{m(r\cos\theta-d)}{\fkuohao{r^2+d^2-2rd\cos\theta}^{\frac{3}{2}}}-
            \frac{\kuohao{\frac{a}{d}}^3m\kuohao{r\cos\theta-\frac{a^2}{d}}}{\fkuohao{r^2+\kuohao{\frac{a^2}{d}}^2-2r\kuohao{\frac{a^2}{d}}\cos\theta}^{\frac{3}{2}}} \\
            &=
            \frac{m(r\cos\theta-d)}{\fkuohao{r^2+d^2-2rd\cos\theta}^{\frac{3}{2}}}-
            \frac{m\kuohao{r\cos\theta-\kuohao{\frac{a}{d}}^2d}}{\fkuohao{a^2+\kuohao{\frac{r}{a}}^2d^2-2rd\cos\theta}^{\frac{3}{2}}}
        \end{aligned}
    }
    则在边界上有
    \equa{
        \begin{aligned}
            B_n
            &=-\left.\pd{\varphi_m}{r}\right|_{r\to a}\\
            &=
            -\frac{m\cos\theta}{\fkuohao{a^2+d^2-2ad\cos\theta}^{\frac{3}{2}}}
            +\frac{3}{2}m\kuohao{a\cos\theta-d}\frac{2a-2d\cos\theta}{\fkuohao{a^2+d^2-2ad\cos\theta}^{\frac{5}{2}}}\\
            &\quad
            +\frac{m\cos\theta}{\fkuohao{a^2+d^2-2ad\cos\theta}^{\frac{3}{2}}}
            -\frac{3}{2}m\kuohao{a\cos\theta-\kuohao{\frac{a}{d}}^2d}\frac{2\kuohao{\frac{a}{d}}^2a-2d\cos\theta}{\fkuohao{a^2+d^2-2ad\cos\theta}^{\frac{5}{2}}}\\
            &=3m\frac{\kuohao{a\cos\theta-d}\kuohao{a-d\cos\theta}-\kuohao{\frac{a}{d}}\kuohao{a-d\cos\theta}\kuohao{\frac{d}{a}}\kuohao{a\cos\theta-d}}{\fkuohao{a^2+d^2-2ad\cos\theta}^{\frac{5}{2}}}\\
            &=0
        \end{aligned}
    }
    满足边界条件(\ref{3.2_bianj}),由唯一性定理可知,该镜像即唯一解。
    

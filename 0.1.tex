\section{证明$\varepsilon_{ijk}$是三维空间的三阶“物理”张量。}

    \subsection{数学准备:定义、结论与说明}

    课件上对“物理”张量的定义是:在空间的转动或反演下不变。
    
    那么我们先说结论:\nota{\ep}\textbf{不是}三维空间的三阶“物理”张量,它属于赝张量。也就是说,\nota{\ep}在空间的转动操作下不变,而在空间反演操作下变号。下面我们将说明这一点。
    
    在此之前,在三维空间中定义三阶赝张量为:如果三维空间中的一个量$\mathbf{\varepsilon}$对于基变换
    \begin{equation}
        \mathbf{\hat{e'}}_{i'} = A_{i'i} \bm{e}_i ,
        \label{0.1_基}
    \end{equation}
    满足分量赝协变关系\footnote{使用爱因斯坦求和约定。}
    \begin{equation}
        \varepsilon'_{i'j'k'} = 
        \begin{cases}
            A_{i'i}A_{j'j}A_{k'k} \varepsilon_{ijk} & \text{空间旋转},\\
            - A_{i'i}A_{j'j}A_{k'k} \varepsilon_{ijk} & \text{空间反演},
        \end{cases}
        \label{0.1_赝}
    \end{equation}
    则$\mathbf{\varepsilon}$叫做三阶赝张量。
    
    \subsection{方法一:利用\nota{\ep}的定义}
    
        这种方法只能只能证明\nota{\ep}不是张量,但是却不能严格证明它是赝张量。
        
        我们按照\nota{\ep}的一种定义:
        
            \begin{equation}
                \varepsilon_{ijk} = \varepsilon_{jki} = \varepsilon_{kij} = -\varepsilon_{jik} = -\varepsilon_{ikj} = -\varepsilon_{kji}, \label{0.1_对称}
            \end{equation}
            \begin{equation}
                \varepsilon_{123} = 1 \label{0.1_1}.
            \end{equation}
        
        我们用反证法。如果\nota{\ep}是张量,则由指标反对称关系的协变性,(\ref{0.1_对称})在基变换下是必然成立的,即
        \begin{equation}
                \varepsilon'_{i'j'k'} = \varepsilon'_{j'k'i'} = \varepsilon'_{k'i'j'} = -\varepsilon'_{j'i'k'} = -\varepsilon'_{i'k'j'} = -\varepsilon'_{k'j'i'} .
                \label{0.1_对称'}
        \end{equation}
        下面我们需要讨论(\ref{0.1_1})是否会随着基变换而变化。
        
        我们来计算张量$\varepsilon'_{i'j'k'}$的二阶范数,由正交变换的性质可知
        \begin{equation}
            \left \| \varepsilon'_{i'j'k'}  \right \| _2 = 
            \left \| \varepsilon_{ijk}  \right \| _2,
        \end{equation}
        再结合(\ref{0.1_对称})(\ref{0.1_对称'})立即得到
        \begin{equation}
            (\varepsilon'_{i'j'k'})^2 = 1.
        \end{equation}
        则$\varepsilon'_{i'j'k'}$既可能是$+1$,也可能是$-1$,不能推出(\ref{0.1_1})在基变换下仍然成立。
        
        故,由(\ref{0.1_对称})(\ref{0.1_1})定义的\nota{\ep}不是三维空间中的张量。
    
    \subsection{方法二:利用\nota{\ep}的性质}
    
    \nota{\ep}的性质可以用来定义叉乘,这表述为
    \begin{equation}
        \varepsilon_{ijk} = (\mathbf{
        \hat{e}_i} \times \bm{e}_j)\cdot \bm{e}_k,
        \label{0.1_性质}
    \end{equation}
    直接将基变换关系(\ref{0.1_基})代入(\ref{0.1_性质})可以得到以下两式
    \begin{eqnarray}
        \varepsilon_{i'j'k'} 
        & = & (\bm{e}_{i'} \times \bm{e}_{j'})\cdot \bm{e}_{k'}\\
        & = & \left[(A_{i'i}\bm{e}_i)\times(A_{j'j}\bm{e}_j) \right]\cdot (A_{k'k}\bm{e}_k)\\
        & = & A_{i'i}A_{j'j}A_{k'k}(\bm{e}_i \times \bm{e}_j)\cdot \bm{e}_k,\label{0.1_e1}
    \end{eqnarray}
    \begin{equation}
        (\bm{e}_{i'} \times \bm{e}_{j'})\cdot \bm{e}_{k'} = 
        \begin{cases}
            (\bm{e}_i \times \bm{e}_j)\cdot \bm{e}_k & \text{空间旋转},\\
            - (\bm{e}_i \times \bm{e}_j)\cdot \bm{e}_k & \text{空间反演}.\label{0.1_e2}
        \end{cases}
    \end{equation}
    由(\ref{0.1_e1})(\ref{0.1_e2})就可以回到定义(\ref{0.1_赝})。故此证明了\nota{\ep}是三维空间中的三阶赝张量。
    
\section{写出并矢$\bm{e}_3\bm{e}_1$和$\bm{e}_1\bm{e}_3$在笛卡尔坐标系的分量矩阵。}

    $\bm{e}_3\bm{e}_1$的分量为
    \begin{equation}
        \kuohao{\bm{e}_3 \bm{e}_1}_{ij} = ((\bm{e}_3\bm{e}_1)\cdot \bm{e}_i)\cdot \bm{e}_j.
    \end{equation}
    由此得$\bm{e}_3\bm{e}_1$在笛卡尔坐标系中的分量矩阵为
    \begin{equation}
        \bm{e}_3\bm{e}_1 = \bm{e}_3\otimes\bm{e}_1 \longrightarrow  
        \begin{bmatrix}
            0 \\
            0 \\
            1
        \end{bmatrix}
        \begin{bmatrix}
            1  & 0 & 0
        \end{bmatrix} = 
        \begin{bmatrix}
            0  & 0 & 0\\
            0  & 0 & 0\\
            1  & 0 & 0
        \end{bmatrix}.
    \end{equation}
    同理
    \begin{equation}
        \bm{e}_1\bm{e}_3 = \bm{e}_1\otimes\bm{e}_3 \longrightarrow  
        \begin{bmatrix}
            1 \\
            0 \\
            0
        \end{bmatrix}
        \begin{bmatrix}
            0  & 0 & 1
        \end{bmatrix} = 
        \begin{bmatrix}
            0  & 0 & 1\\
            0  & 0 & 0\\
            0  & 0 & 0
        \end{bmatrix}.
    \end{equation}
    两式中均使用了“$\longrightarrow $”表明张量和它的坐标只是对应而非相等关系。

\thispagestyle{plain}
\vspace*{\fill}
    \begin{center}
        \begin{minipage}{0.75\textwidth}
            \begin{spacing}{1.5}
            
                \suojin
                这本书其实是中山大学李志兵、侯玉升老师于2022年所授的电动力学课的历次课程作业题目集合。题目是两位老师出的,题解是我写的。
                
                \suojin
                本书过半数的题解,都或多或少相对于普通的解答而言是有“创新”之处的。这里的“创新”,有的是给出了更深刻的物理图像,有的是解题思路的取巧,有的是更严谨完备的表述,有的是更直观的表述形式,有的是更方便的求解过程等。我觉得写得都挺易懂的。希望读者随便翻开一页,都有所启发。
                
                \suojin
                主要的参考书是郭硕鸿《电动力学(第三版)》,书上的少量错误,我在脚注中指出了。还有一些同学们容易出错的地方,或者容易误解的地方,我也在脚注中指出了。
                
                \suojin
                以普遍理性而言,这些题解不会有原则性的错误,有一些typo是在所难免的。如果发现任何错误,都可以发邮件到\url{yuyk6@mail2.sysu.edu.cn}联系我修改。或者(更推荐)自行到本书的开源项目\url{https://github.com/YinkaiYu/Electrodynamics_yyk}中修改。
                
                \suojin
                将作业写成电子版最初是因为李志兵老师说的仪式感。一开始觉得有趣就这么做了,坚持到底是因为强迫症,因此最后集结成册,只是一种整理欲。这本书应该不会对考试有太大帮助,很多题考试应该不考。本书不是为考试而作。
                
                \suojin
                对本书提供帮助的除了两位老师之外,还有许多同学:王誉晨、罗俊平、赖睿然、莫梁虹、曾植、李小康。他们对题解提供了一定程度的参考。
                
                \suojin
                一般来讲,这本书(至少这个版本)的有效期不太长,后续肯定会修改,或者这本书没啥用了。如果给它定个保质期,就2024年吧。
                
                \suojin
        
                \rightline{余荫铠}
                \rightline{\today~于康乐园}
            \end{spacing}
        \end{minipage}
    \end{center}
\vspace*{\fill}

\question{在狭义相对论中粒子的动量为\nota{\p =m\v},
其中\nota{\v=\dd{\x}{t},m=\ga m_0}; 粒子的动能\nota{T=mc^2-m_0c^2}; 粒子受到的力\nota{\bm{F}}等于其动量变化率。请证明动能变化定理\nota{\d T = \f\d\x}}

    由质能关系和相对论动力学关系
    \footnote{在相对论的运算中简记\equa{\be=v/c\qquad\ga=\frac{1}{\sqrt{1-\be^2}}}}
    \equa{W=mc^2=\ga m_0c^2}
    \equa{W^2=m_0^2c^4+p^2c^2}
    可以证明
    \begin{multline}
        \lhs
        =\d\kuohao{W-m_0c^2}
        =\d W
        =\d\sqrt{m_0^2c^4+p^2c^2}
        =\frac{c^2}{W}\p\cdot\d\p \\
        =\frac{c^2}{mc^2}m\v\cdot\d\p
        =\v\cdot\p
        =\dd{\x}{t}\cdot\d\p
        =\d\x\dd{\p}{t}
        =\f\cdot\d\x=\rhs
    \end{multline}

\question{请证明洛伦兹力密度(作用在单位体积带电物质的洛伦兹力)\nota{\bm{f}=\rho\e+\j\times\b}是一个狭义相对论协变矢量的前三个分量。求出此协变量的第四分量。}

    定义一个洛伦兹协变矢量
    \footnote{
        记洛伦兹协变时空坐标
        \equa{x^\mu=
        \begin{pmatrix}
         \x & \i ct
        \end{pmatrix}\trans
        }
        相应的度规为
        \equa{g_{\mu\nu}=\diag{1&1&1&1}}
        % x-boost洛伦兹变换张量
        % \equa{
        %     \zhang{\Lambda}{\mu}{\nu}=
        %     \mat{
        %         \gamma & -\i\beta\gamma &   &   \\
        %         \i\beta\gamma & \gamma &   &   \\
        %           &   & 1 &   \\
        %           &   &   & 1 \\
        %     }
        % }
        % 特别地,此时有
        % \equa{\Lambda\trans=\Lambda\inv\label{6.5_正交}}
    }
    \equa{f_\mu=\zhang{F}{\mu\nu}{}J_\nu\label{6.5_2.1}}
    由于\nota{\zhang{F}{\mu\nu}{}}是洛伦兹协变张量,\nota{J_\nu}是洛伦兹协变矢量,而\nota{f_\mu}由\nota{\zhang{F}{\mu\nu}{}}与\nota{J_\nu}缩并而来,故显然这样定义的\nota{f_\mu}是洛伦兹协变的矢量。
    
    下面我们证明\nota{\bm{f}}的三个分量就是\nota{f_\mu}的前三项。
    我们在实空间表象中
    \footnote{我们用\nota{\backsimeq}表示张量与其在表象下的坐标的对应。}
    将\neq{6.5_2.1}的分量坐标全部写出,即
    \equa{
        \mat{f_1\\f_2\\f_3\\f_4\\}
        =
        \mat{
            0 & B_3 & -B_2 & -\i E_1/c \\
            -B_3 & 0 & B_1 & -\i E_2/c \\
            B_2 & -B_1 & 0 & -\i E_3/c \\
            \i E_1/c & \i E_2/c & \i E_3/c & 0 \\
        }
        \mat{J_1\\J_2\\J_3\\\i c\rho\\}
        =
        \mat{
            J_2 B_3 - J_3 B_2 + \rho E_1 \\
            J_3 B_1 - J_1 B_3 + \rho E_2 \\
            J_1 B_2 - J_2 B_1 + \rho E_3 \\
            \i J_1E_1/c + \i J_2E_2/c + \i J_3E_3/c \\
        }
        \label{6.5_2.2}
    }
    于是可以看出
    \equa{
        \bm{f}=\rho\e+\j\times\b\backsimeq
        \mat{
            J_2 B_3 - J_3 B_2 \\
            J_3 B_1 - J_1 B_3 \\
            J_1 B_2 - J_2 B_1 \\
        }
        + \rho
        \mat{E_1\\E_2\\E_3\\}
        =
        \mat{f_1\\f_2\\f_3\\}
    }
    这就证明了洛伦兹力密度(作用在单位体积带电物质的洛伦兹力)\nota{\bm{f}=\rho\e+\j\times\b}是狭义相对论协变矢量\nota{f_\mu}的前三个分量。
    
    由\neq{6.5_2.2}知,\nota{f_\mu}的第四个分量为
    \equa{f_4=\frac{\i}{c}\j\cdot\e}
